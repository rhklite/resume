\cvsection{Experiences}
\begin{cventries}
	\cventry
	{Researcher, Master Thesis}
	{Autonomous System and Biomechatronics Lab}
	% {Toronto, Ontario}
	{Sep. 2018 - Aug. 2021}
	{}
	{
		\begin{cvitems}
			\item {
				\textbf{Deep Learning} Developed and published a novel sim-to-real transfer pipeline for robot navigation in Pytorch, achieved 87\% real world success rate given a 90\% simulation success rate
			}
			\item {
				\textbf{Development} Developed a ROS based decentralized software and hardware robot architecture using C++ and Python
			}
			\item {
				\textbf{Localization} Implemented LiDAR and visual SLAM on a mobile robot for real time pose estimation
			}
			\item {
				\textbf{Control} Designed and optimized a cascade PID controller for global position and wheel control in rough terrain
			}
			\item {
				\textbf{Analysis} Analyzed the pipeline with autonomous navigation experiments, comparison studies, and ablation studies
			}
			% \item {
			% 	\textbf{Hardware} Enhanced a robot with auxilliary computing units and sensors with components designed using SolidWorks
			% }
			% \item {
			% 	\textbf{Publication} Published in 1) IEEE Robotics and Automation Letters and 2) International Conference on Intelligent Robots and Systems
			% }
		\end{cvitems}
	}
	\cventry
	{Support Researcher, Autonomous Driving Division}
	{Huawei Noah’s Ark Lab}
	% {Toronto, Ontario}
	{May. 2020 - Jan 2021}
	{}
	{
		\begin{cvitems}
			\item {
				\textbf{Path Planning} Developed, published, and patented a novel spatial constraint generation algorithm for autonmous driving in python
			}
			\item {
				\textbf{Simulation} Engaged in an autonmous driving simulator development using real-world datasets
			}
			% \item{
			% 	\textbf{Publications} Paper accepted to IROS 2021 Conference; Provisinal Patent Application Number: 63/108,348
			% }
		\end{cvitems}
	}
	\cventry
	{Head Teaching Assistent}
	{MIE443 Mechatronics Systems: Design \& Integration}
	% {Toronto, Ontario}
	{Jan. 2018 - Apr. 2020}
	{}
	{
		\begin{cvitems}
			\item {
				\textbf{Lecture} Lectured on ROS based robot navigation and SLAM methods
			}
			\item {
				\textbf{Mentorship} Guided students on ROS based autonomous robot algorithm development, vision sensor, and OpenCV
			}
			% \item \textbf{\href{https://www.mie.utoronto.ca/congratulations-2019-20-teaching-assistant-award-winners-lap-tak-chu-richard-hu-behzad-khamidehi-ben-leung-and-khalil-sidawi/}{Award}} Recipient of UofT MIE 2019-20 Teaching Assistant Award
		\end{cvitems}
	}
	\cventry
	{Researcher, Pico-Scale Hydro Turbine Design}
	{Water and Energy Research Laboratory}
	% {Toronto, Ontario}
	{Jan. 2018 - Sep. 2018}
	{}
	{
		\begin{cvitems}
			\item {\textbf{Mechanical} Designed a variable guide vane for pico-scale hydro turbine using SolidWorks}
			\item {\textbf{Analysis} Evaluated the guide vane failure mode with fluid pressure test, mechanical stress test, and finite element analysis}
			\item {\textbf{Development} Prototyped the turbine and an experiment pipeline using Arduino, SLA 3D printing and machining techniques}
		\end{cvitems}
	}
	\cventry
	{Mechanical Engineer Intern, Novasight Hybrid System}
	{Conavi Medical} 
	% {Toronto, Ontario}
	{May. 2016 - Aug. 2017}
	{}
	{
		\begin{cvitems}
			\item {
				\textbf{Analysis} Investigated potential design hazards and risks of catheter rotary assembly
			}
			\item {
				\textbf{Manufacturing} Streamlined an efficient assembly and calibration work instruction for intravascular catheter}
			\item {
				\textbf{Organization} Established an inventory system with full traceability for FDA 510k submission validation}
			\item {
				\textbf{Management} Directed technical design reviews with senior leadership, accelerated the exit of the project phase}
			\item {
				\textbf{Mechanical} Designed imaging and rotary assembly for a intravascular catheter using MATLAB and SolidWorks
			}
		\end{cvitems}
	}
	\cventry
	{Researcher}
	{Multiphase Flow and Spray Systems Lab} 
	% {Toronto, Ontario}
	{Jun. 2015 - Sep. 2015}
	{}
	{
		\begin{cvitems}
			\item {
				\textbf{Development} Developed Arduino based camera to fluid pipeline synchronization system to speed up data collection by 85\%
			}
			\item {
				\textbf{Analysis} Classified 13 novel air-fluid impingement shatter pattern using statistical analysis}
		\end{cvitems}
	}  
\end{cventries}
%%%%%%%%%%%%%%%%%%%%%%%%%%%%%%%%%%%%%%
%     Other Experiences
%%%%%%%%%%%%%%%%%%%%%%%%%%%%%%%%%%%%%%
